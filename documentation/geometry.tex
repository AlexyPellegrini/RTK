\documentclass{article}
\usepackage[textwidth=16cm,textheight=24.7cm,top=2.5cm,headheight=15pt]{geometry}

\usepackage{color}

\newcommand{\question}[1]{\textcolor{red}{--- #1 ---}}

\newcommand{\biec}{\begin{quote}\begin{small}}
\newcommand{\eiec}{\end{small}\end{quote}}

\usepackage{amsmath}
\usepackage{bm}
\newcommand{\degree}{\ensuremath{^\circ}}

\begin{document}

\title{RTK geometry v1}
\author{Simon Rit}

\maketitle

\section{Purpose}

The purpose of this document is to describe the geometry format used in RTK to relate a tomography to projection images. There is currently only one geometry format, ThreeDCircularProjectionGeometry.

\section{Units}

\begin{itemize}
 \item Degrees are used to store angles in the geometry objects. Angles are wrapped between 0\degree and 360\degree.
 \item No unit is enforced for distances but it is the responsibility of the user to have a consistent unit for all distances (pixel and voxel spacings, geometry parameters...). Millimeters are typically used in ITK and DICOM.
\end{itemize}

\section{Image coordinate system}

An \verb+itk::Image<>+ contains information to convert voxel indices to physical coordinates using its members \verb+m_Origin+, \verb+m_Spacing+ and \verb+m_Direction+. Voxel coordinates are not used in RTK except for internal computation. The conversion from voxel index coordinates to physical coordinates and the dimensions of the images are out of the scope of this document.

\section{Fixed coordinate system}

The fixed coordinate system $(x,y,z)$ in RTK is the coordinate system of the tomography with the isocenter at the origin $(0,0,0)$.

\section{ProjectionGeometry$<$TDimension$>$}

This is the mother class for relating a TDimension-D tomography to a (TDimension-1)-D projection image. It holds a vector of (TDimension)x(TDimension+1) projection matrices accessible via \verb+GetMatrices+. The construction of those matrices is geometry dependent.

\section{ThreeDCircularProjectionGeometry}

This class is meant to define a set of 2D projection images, acquired with a flat panel along a circular trajectory, around a 3D tomography. The trajectory does not have to be strictly circular but it is assumed in some reconstruction algorithms that the rotation axis is y. The description of the geometry is based on the international standard IEC 61217 which has been designed for cone-beam imagers on isocentric radiotherapy systems but it can be used for any 3D circular trajectory. The fixed coordinate system of RTK and the fixed coordinate system of IEC 61217 are the same.

9 parameters are used per projection to define the position of the source and the detector relatively to the fixed coordinate system. The 9 parameters are set with the method \verb+AddProjection+. Default values are provided for the parameters which are not mandatory. Note that explicit names have been used but this does not necessarily correspond to the value returned by the scanner which can use its own parameterization.

\subsection{Detector orientation}

\paragraph{Initial detector orientation}

With all parameters set to 0, the detector is normal to the z direction of the fixed coordinate system, similarly to the x-ray image receptor in the IEC 61217.

\paragraph{Rotation order} Three rotation angles are used to define the orientation of the detector. The ZXY convention of Euler angles is used for detector orientation where GantryAngle is the rotation around y, OutOfPlaneAngle the rotation around x and InPlaneAngle the rotation around z. These three angles are detailed in the following.

\paragraph{GantryAngle}

Gantry angle of the scanner. It corresponds to $\phi g$ in Section 2.3 of IEC 61217:

\biec
The rotation of the "g" system is defined by the rotation of coordinate axes Xg, Zg by an angle $\phi g$ about axis Yg (therefore about Yf of the "f" system).

An increase in the value of $\phi g$ corresponds to a clockwise rotation of the GANTRY as viewed along the horizontal axis Yf from the ISOCENTRE towards the GANTRY.
\eiec

\paragraph{OutOfPlaneAngle}

Out of plane rotation of the flat panel complementary to the GantryAngle rotation, i.e. with a rotation axis perpendicular to the gantry rotation axis and parallel to the flat panel. It is optional with a default value equals to 0. There is no corresponding rotation in IEC 61217. After gantry rotation, the rotation is defined by the rotation of the coordinate axes y and z about x. An increase in the value of OutOfPlaneAngle corresponds to a counter-clockwise rotation of the flat panel as viewed from a positive value along the x axis towards the isocenter.

\paragraph{InPlaneAngle} 

In plane rotation of the 2D projection. It is optional with 0 as default value. If OutOfPlaneAngle equals 0, it corresponds to $\theta r$ in Section 2.6 of IEC 61217:


\biec
The rotation of the "r" system is defined by the rotation of the coordinate axes Xr, Yr about Zr (parallel to axis Zg) by an angle $\theta r$.

An increase in the value of angle $\theta r$ corresponds to a counter-clockwise rotation of the X- RAY IMAGE RECEPTOR as viewed from the RADIATION SOURCE.
\eiec

\paragraph{Rotation matrix}

The rotation matrix in homogeneous coordinate is then (constructed with\\ \verb+itk::Euler3DTransform<double>::ComputeMatrix()+ with opposite angles because we rotate the volume coordinates instead of the scanner):

$$
\begin{array}{lcll}
  M_R & = & & %
  \begin{pmatrix}
    \cos(-InPlaneAngle) & -\sin(-InPlaneAngle) & 0 & 0\\
    \sin(-InPlaneAngle) & \cos(-InPlaneAngle) & 0 & 0\\
    0 & 0 & 1 & 0\\
    0 & 0 & 0 & 1
  \end{pmatrix} \\
  \\ & & \times & %
  \begin{pmatrix}
    1 & 0 & 0 & 0\\
    0 & \cos(-OutOfPlaneAngle) & -\sin(-OutOfPlaneAngle) & 0\\
    0 & \sin(-OutOfPlaneAngle) & \cos(-OutOfPlaneAngle) & 0\\
    0 & 0 & 0 & 1
  \end{pmatrix} \\
  \\ & & \times & %
  \begin{pmatrix}
    \cos(-GantryAngle) & 0 & \sin(-GantryAngle) & 0 \\
    0 & 1 & 0 & 0 \\
    -\sin(-GantryAngle) & 0 & \cos(-GantryAngle) & 0 \\
    0 & 0 & 0 & 1
  \end{pmatrix}
\end{array}
$$

\subsection{Source position}

The source position is defined with respect to the isocenter with three parameters, SourceOffsetX, SourceOffsetY and SourceToIsocenterDistance. (SourceOffsetX,SourceOffsetY,-SourceToIsocenterDistance) are the coordinates of the source in the rotated coordinated system. In IEC 61217, SourceToIsocenterDistance is the RADIATION SOURCE axis distance, SAD. SourceOffsetX and SourceOffsetY are optional and zero by default.

\subsection{Detector position}

The detector position is defined with respect to the source with three parameters: ProjectionOffsetX, ProjectionOffsetY and SourceToDetectorDistance. (ProjectionOffsetX,ProjectionOffsetY,SourceToIsocenterDistance-SourceToDetectorDistance) are the of the coordinates of the detector origin in the rotated coordinated system. In IEC 61217, SourceToDetectorDistance is the RADIATION SOURCE to IMAGE RECEPTION AREA distance, SID. ProjectionOffsetX and ProjectionOffsetY are optional and zero by default.

\subsection{Final matrix}

Each matrix, accessible via \verb+GetMatrices+, is constructed with:

$$
\begin{array}{lcll}
  M_P & = & & %
  \begin{pmatrix}
    1 & 0 & SourceOffsetX-ProjectionOffsetX  \\
    0 & 1 & SourceOffsetY-ProjectionOffsetY  \\
    0 & 0 & 1
  \end{pmatrix} %
  \\ \\ & & \times & %
  \begin{pmatrix}
    -SourceToDetectorDistance & 0 & 0 & 0  \\
    0 & -SourceToDetectorDistance & 0 & 0  \\
    0 & 0 & 1 & -SourceToIsocenterDistance
  \end{pmatrix} %
  \\ \\ & & \times & %
  \begin{pmatrix}
    1 & 0 & 0 & -SourceOffsetX  \\
    0 & 1 & 0 & -SourceOffsetY  \\
    0 & 0 & 1 & 0 \\
    0 & 0 & 0 & 1
  \end{pmatrix} %
  \\ \\ & & \times & %
  M_R
\end{array}
$$

\subsection{XML file}

ThreeDCircularProjectionGeometry can be saved and loaded from an XML file. If the parameter is equal to the default value for all projections, it is not stored in the file. If it is equal for all projections but different from the default value, it is stored once. Otherwise, it is stored for each projection. The matrix is given for information. It is read and checked to be consistent with the parameters but a manual modification of the file must consistently modify both the parameters and the matrix. An example is given hereafter:

\begin{verbatim}
<?xml version="1.0"?>
<!DOCTYPE RTKGEOMETRY>
<RTKThreeDCircularGeometry version="0">
  <SourceToIsocenterDistance>1000</SourceToIsocenterDistance>
  <SourceToDetectorDistance>1536</SourceToDetectorDistance>
  <Projection>
    <GantryAngle>271.847274780273</GantryAngle>
    <ProjectionOffsetX>-117.056503295898</ProjectionOffsetX>
    <ProjectionOffsetY>-1.01195001602173</ProjectionOffsetY>
    <Matrix>
          -166.5093078829                   0   -1531.42837748039   -117056.503295898
        -1.01142410874151               -1536  0.0326206557691505   -1011.95001602173
       -0.999480303105996                   0  0.0322354417240802               -1000
    </Matrix>
  </Projection>
  <Projection>
    <GantryAngle>271.852905273438</GantryAngle>
    <ProjectionOffsetX>-117.056831359863</ProjectionOffsetX>
    <ProjectionOffsetY>-1.01187002658844</ProjectionOffsetY>
    <Matrix>
        -166.660129424325                   0   -1531.41199650136   -117056.831359863
        -1.01134095059569               -1536  0.0327174625589984   -1011.87002658844
       -0.999477130482326                   0  0.0323336611415466               -1000
    </Matrix>
  </Projection>
</RTKThreeDCircularGeometry>
\end{verbatim}


\end{document}
